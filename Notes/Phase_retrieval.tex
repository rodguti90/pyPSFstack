\documentclass[reprint,aps,pra,superscriptaddress,
amsmath,amssymb]{revtex4-1}
\usepackage{mathtools}
\usepackage{graphicx}
\usepackage{booktabs,multirow}
\usepackage{braket}
\usepackage{textcomp}
\usepackage{yfonts}
\usepackage[
  bookmarks=false,
  %bookmarks=true,
  colorlinks,
  linkcolor=blue,
  urlcolor=blue,
  citecolor=blue,
  plainpages=false,
  pdfpagelabels,
  final,
  breaklinks=true
]{hyperref}

\begin{document}
\newcommand{\pd}[2]{\frac{\partial #1}{\partial #2}} 
% for partial derivatives
\newcommand{\td}[2]{\frac{d #1}{d #2}} 

\newcommand{\bs}{\boldsymbol}
\newcommand{\bt}{\textbf}
\newcommand{\sech}{\text{sech}}
\newcommand{\erfc}{\text{erfc}}
\newcommand{\bse}{\begin{subequations}}
\newcommand{\ese}{\end{subequations}}
\newcommand{\im}{\text{i}}
\newcommand{\ud}[0]{\mathrm{d}}
\newcommand{\norm}[1]{\left\lVert#1\right\rVert}
\newcommand{\op}{\widehat}

\graphicspath{{figures/},{../figures/}} % specifies the path for the figures
\allowdisplaybreaks

\title{Characterizing the polarization aberrations in widefield fluorescence microscopy using phase retrieval}

\author{R. Guti\'{e}rrez-Cuevas}
\email{rodrigo.gutierrez-cuevas@fresnel.fr}
\affiliation{Aix Marseille Univ, CNRS, Centrale Marseille, Institut Fresnel, UMR 7249, 13397 Marseille Cedex 20, France}
\affiliation{Institut Langevin, ESPCI Paris, Université PSL, CNRS, 
75005 Paris, France}

\date{\today}

\begin{abstract}
Here we present the theory and the different strategies for retrieving the 
pupil of in fluorescence microscopy where the effects of birefringence play 
an  important role. We also study the effect of separating the signal into 
different polarization channels.
\end{abstract}


\pacs{}

\maketitle

%----------------------------------------------------------------------------
%---------------------------Introduction-------------------------------------
%----------------------------------------------------------------------------

\section{Intro}

General intro about fluorescence microscopy, SMOLM and PSF engineering.
Finish by saying something along the lines of:
However, any aberration or misalignment in the system can affect the 
final shape of the PSFs and thus lead to an inaccurate estimate of the 
parameters. 

A common solution to this problem is to perform a set of calibration 
measurements with a known source and at varying focal planes 
\cite{hanser2003phase,hanser2004phase}. From 
these measurements and an accurate model for the propagation of the light
emitted by the source to the camera, the aberrations can be estimated 
through phase retrieval algorithms. Initially, only scalar models with point 
sources were used which allowed the successful use of both iterative and 
nonlinear optimization routines for estimating the aberrations 
\cite{hanser2003phase,hanser2004phase,clark2012microscope}. However, 
more accurate models \cite{thao2019phase,ferdman2020vipr} take into 
account the vectorial nature of light which is essential to describe 
the propagation of the emitted radiation through the 
interface between the suspension medium and the immersion liquid used for 
the high-NA microscope objective \cite{novotny2006principles}. Moreover,
it is also necessary to consider that the most common sources, 
such as fluorescent nanobeads, emit 
incoherent light and their size can thus lead to a noticeable blurring of 
the PSFs.
Given these more accurate assumptions, 
iterative algorithms cannot be directly applied and have to be adapted which 
can make them unstable, although some success has been demonstrated 
\cite{thao2019phase}. A more natural approach is to use nonlinear optimization
\cite{fienup1982phase,ferdman2020vipr}
since it provides us with the freedom to incorporate other unknown parameters
into the estimation, such as the photobleaching amplitudes or the background 
illumination.

Up until now, all works have assumed a scalar pupil to characterize 
the aberrations or to design new PSFs. However, SMOLM requires the use 
of birefringent elements to encode the orientation information of the 
emitting dipole
into the shape of the two polarization components of the PSF. 
Therefore, it is important to update the model of propagation for an accurate
characterization or design of SMOLM systems. In this work, we propose a 
characterization technique and algorithm to estimate the polarization aberrations 
of the system \cite{hansen1988overcoming} as well as other unknown (or poorly known) parameters. 
This is done by applying a nonlinear optimization algorithm to a physical model
where the aberrations (or widow we want to design) are now represented by a Jones
matrix. The implementation is done with the 
neural network framework \texttt{PyTorch} \cite{} which performs automatically
all the gradient computation \cite{}. Additionally, we show that, in general,
it is necessary to introduce polarization diversity in the measurements 
in order to properly characterize the polarization response. This is similar 
to the introduction of phase diversity by taking images at various focal planes.
Additionally, we implement the most useful method for taking into account the 
blurring to the size of the nanobead discussed in \cite{}.
It should also be noted that all the results obtained in this work 
were achieved with the software \texttt{torchPSFstack} which is freely 
accessible at \cite{}.

% This is the goal of this work 
% where we define a phase retrieval algorithm to retrieve the Jones matrix at 
% the BFP that accounts for all the imperfections in the system. The approach 
% is base on the assumptions of an incoherent point source, which is the case 
% for fluorescing nanobeads used for most characterizations, and the use of 
% both phase and polarization diversities. The phase diversity is the standard 
% approach of taking several images across several focal planes and the 
% polarization diversity is obtained by using a polarization analyzer. Moreover, 
% given several measurements across the field it will also be possible to 
% determine the field dependent polarization aberrations 
% \cite{zheng2013characterization}.


% In general terms the goal of phase retrieval is to estimate the phase of a 
% field given a set of intensity measurements. In imaging applications, the 
% spatial distribution of a given object can then be reconstructed from the 
% complete information of the field. The different intensity measurements can 
% be performed at different focal or image planes, or even under varying 
% illumination conditions. In the particular case of fluorescence microscopy 
% \cite{moerner2003methods}, the objective is to retrieve the field 
% distribution at the back-focal plane (BFP) from intensity measurements of 
% the point-spread function at varying focal planes 
% \cite{hanser2003phase,hanser2004phase}. In this case it might be more 
% adequate to call the task pupil retrieval since we are looking for both 
% the phase and amplitude at the BFP. But, following common practice we will 
% use them interchangeably since the field at the BFP is linked to the one 
% at the image plane via a scaled Fourier transformation so, in principle, 
% complete knowledge from one entails complete knowledge of the other.

% The most common and best known method for phase retrieval is the iterative 
% Gerchberg-Saxton (GS) algorithm which is an iterative approach performed by 
% Fourier transforming back and forth from the BFP to the image plane 
% \cite{shechtman2015phase}. Depending on the particular setup and application, 
% it might also be possible to use intensity measurements at the BFP and at 
% various focal or image planes which can help the convergence of the 
% algorithm. One great advantages of the GS algorithm is that it is fast 
% and easy to implement. Moreover, it provides an estimate for the intensity 
% distribution at the Fourier plane (if it is not given). However, it is not 
% directly applicable to every situation and might not provide the best 
% estimate. This is the reason why phase retrieval algorithms have moved 
% away from the iterative approach of the GS algorithm and into the domain 
% of nonlinear optimization routines \cite{fienup1982phase}. 
% This latter approach provides a lot of freedom since any experimental 
% parameter that is not perfectly known can be estimated. 

% In fluorescence microscopy, phase retrieval algorithms are used to estimate 
% the aberrations and imperfections of the experimental setup. A good estimate 
% is essential since it can have drastic effects on the shape of the intensity 
% distribution of the PSF which is used to encode additional information about 
% the emitter, such as its longitudinal position 
% \cite{moerner2003methods,curcio2020birefringent}. Both iterative and 
% nonlinear optimization routines have been used successfully 
% \cite{hanser2003phase,hanser2004phase,clark2012microscope}. The standard 
% approach has been to assume a fully scalar model and it is only recently that 
% the true vector nature of the dipolar emitters has been taken into account  
% \cite{thao2019phase,ferdman2020vipr}. The vector nature is essential to 
% accurately model the propagation of the emitted radiation through the 
% interface between the suspension medium and the immersion liquid used for 
% the high-NA microscope objective. 

% Lately, the PSF shaping technique has been extended to encode the 
% orientation of the emitting dipoles by translating the polarization 
% distribution of the dipolar radiation into the shape of the PSF  
% \cite{curcio2020birefringent} . This is inevitably done by polarization 
% optics and birefringent element which have and are more sensitive to 
% polarization aberrations. It is therefore essential to take them into 
% account for an accurate modeling of the PSFs. This is the goal of this work 
% where we define a phase retrieval algorithm to retrieve the Jones matrix at 
% the BFP that accounts for all the imperfections in the system. The approach 
% is base on the assumptions of an incoherent point source, which is the case 
% for fluorescing nanobeads used for most characterizations, and the use of 
% both phase and polarization diversities. The phase diversity is the standard 
% approach of taking several images across several focal planes and the 
% polarization diversity is obtained by using a polarization analyzer. Moreover, 
% given several measurements across the field it will also be possible to 
% determine the field dependent polarization aberrations 
% \cite{zheng2013characterization}.



\section{Modeling the PSFs for fluorescence microscopy}

\subsection{Field at the back-focal plane}

The first step is to have the most accurate model for the propagation 
of light through our microscope. As shown in Fig.~\ref{fig:aplanatic}, 
the source is embedded in a medium of index of refraction $n_i$ at a 
distance $d_{cs}>0$ from the interface with immersion oil of index of 
refraction $n_f$, the interface is given by the coverslip which is assumed
to be index matched to the immersion oil. For a dipolar source, the green 
tensor at the back-focal plane (BFP) of a high-NA objective can be 
computed through Richards-Wolf diffraction 
theory for aplanatic systems \cite{richards1959electromagnetic,
novotny2006principles}. An important step in this calculation is to  
take into account the interface between the two media since it introduces 
spherical aberration and the coupling of evanescent waves when $n_f > n_i$,
known as supercritical angle fluorescence (SAF), 
which  can contribute to more than 
half of the radiation reaching the detector \cite{hellen1987fluorescence,
axelrod2001total,axelrod2013evanescent}.

% \begin{figure}
%   \centering
%   \includegraphics[width=.9\linewidth]{aplanatic.pdf}
%   \caption{\label{fig:aplanatic} . }
%   \end{figure}

Following \cite{lieb2004single,novotny2006principles} and assuming that 
light propagates along the positive $\hat z$ direction, the Green tensor 
at the BFP,
\begin{align} \label{eq:G0}
\bt G_0 (\bt u) = &   e^{-\im k n_f \bs \rho_o \cdot \bt u } 
\exp \left[\im k n_i d_\text{cs} 
\sqrt{1 - \left(\frac{n_f u}{n_i} \right)^2 } \right]  \nonumber \\
& \times \exp \left(\im k n_f z_f \sqrt{1 - u^2} \right) \bt g (\bt u) ,
\end{align}
where 
\begin{align}
\bt g (\bt u) =
\left(
\begin{array}{ccc}
g_{xx}  (\bt u)& g_{xy}  (\bt u)& g_{xz}  (\bt u) \\
g_{yx}  (\bt u)& g_{yy}  (\bt u)& g_{yz} (\bt u)
\end{array}
\right)
\end{align}
with the explicit form of  its components given in the Appendix,
$\bs \rho_o =(x_o,y_o)$  denotes the transverse location of the dipole, 
$k=2 \pi/\lambda$ is the wavenumber with $\lambda$ being the wavelength,
and 
$z_f$ is the location of the focal plane from the interface, $z_f<0$ 
($z_f>0$) if the focal plane is in the medium with index of refraction 
$n_i$ ($n_f$). The vector 
$\bt u = (u_x, u_y)$ denotes the normalized coordinates at the BFP, 
see Fig.~\ref{fig:aplanatic}. The maximum value of $u = \norm{\bt u}$ is 
limited by the NA through $u_\text{max} = \text{NA}/n_f$. 
The matrix elements $g_{ij}$ include the effect of the Fresnel coefficients 
of the boundary and depend on $\bt u$ but not on the location of the dipole 
or the focal plane.  For a dipole oriented along the unit vector 
$\bs \mu = (\mu_x, \mu_y, \mu_z)$, the electric field distribution at the BFP 
is given by 
\begin{align}
\bt E_{0} (\bt u) = \bt g (\bt u; \bt r_0) \cdot \bs \mu.
\end{align}
Therefore, the three columns of the Green tensor represent the field 
distribution produced by a dipole along each of the three coordinate axes. 


\subsection{Propagation from the BFP to the image plane}

\begin{figure}
\centering
\includegraphics[width=.9\linewidth]{setup.pdf}
\caption{\label{fig:setup} . }
\end{figure}


In order to encode information about the position and orientation of the 
dipole into the shape of the PSF, it is necessary to include a mask into the 
path of the emitted light at the BFP with the possible help a relay system 
as shown in Fig.~\ref{fig:setup}. 
The most general case is that of a 
birefringent mask represented by a $2\times 2$ Jones matrix \cite{},
\begin{align} \label{eq:jqw}
\bt J_\text{M}(\bt u) =  \left(
\begin{array}{cc}
J_{xx}&J_{xy} \\
J_{yx} & J_{yy}
\end{array}
\right).
\end{align}
Note that a scalar mask case with apodization is regained if $q_j=0$ 
for $j=1,2,3$, and a pure phase mask if also $q_0 =1$.
After multiplying the Green tensor by this Jones matrix, the field  can then 
be propagated form the BFP to the image plane through
\begin{align}
\bt G_\text{IP} (\bs \rho ) = \iint  \bt J_\text{M}(\bt u)  
\cdot \bt G_0  (\pm \bt u) e^{- \im k  \frac{n_f \bs \rho}{M}  \cdot \bt u } \ud \bt u ,
\end{align}
where $M$ is the total magnification of the system. 
Note that the normalized coordinate is defined at the location of the birefringent  mask.
Assuming an incoherent source, the PSF is then given by
\begin{align}
  I_\text{IP} (\bs \rho) =  \norm{\bt G_\text{IP} }^2 
  = \sum_{i=x,y} \sum_{j=x,y,z} |G_{\text{IP},ij}( \bt u)|^2.
\end{align}

\section{Modeling for the phase retrieval}

\subsection{Polarization aberrations}

The decomposition of polarization aberrations by a Jones matrix can be 
done in the following manner
\begin{align} \label{eq:jqw}
  \bt J_\text{A}(\bt u) = e^{\im 2\pi W(\bt u)} \left(
\begin{array}{cc}
q_0(\bt u) + \im q_3(\bt u) &q_2(\bt u) + \im q_1(\bt u) \\
-q_2(\bt u) + \im q_1(\bt u) & q_0(\bt u) - \im q_3(\bt u)
\end{array}
\right),
\end{align}
which allows separating the scalar aberrations, contained 
in $W$ from the vectorial correction given by the $q$'s. 
Since aberrations tend to be described by smooth functions, it generally 
suffices to decompose the various elements of the Jones matrix using the 
Zernike polynomials which constitute a complete basis on the unit disk.
Therefore, we write
\begin{align}
  W (\bt u) = &\sum'_l c_{l}^{(W)} Z_l(\bt u/u_\text{max}) ,\\
  q_{j} (\bt u) = &\sum_l c_{l}^{(j)} Z_l(\bt u/u_\text{max}),
\end{align}
where $j=0,...,3$, and we used a single index notation for the basis. 
Note that $\sum'$ in the expression for $W$ indicates that the terms 
corresponding to piston and defocus should be excluded, this is 
automatically handled by the software \cite{}. The piston term only 
fixes a global phase which cannot be determined from intensity measurements
while the defocus term is redundant with the more accurate defocus parameter
$z_f$ in Eq.~\ref{eq:G0}.
This Zernike expansion is inspired by the Nijboer-Zernike theory 
\cite{janssen2002extended,braat2003extended,braat2005extended} where a 
scalar mask would be separated into real and imaginary parts before 
decomposing in terms of Zernike polynomials. 
Note that other models can be used, such as 
pixel by pixel decomposition, but a Zernike decomposition with enough 
terms should be able to handle most cases found in microscopes. Moreover, 
computationally there is no advantage to considering a pixel decomposition 
since the number of DFT (the most costly operation) would be the same.
It should also  be noted that a scalar mask can easily be modeled by just
setting the coefficients $c^{(j)}_l=0$ for all $l$ and $j=1,2,3$.


\subsection{Best focus and the distance to the coverslip}

In general the position chosen as the nominal focus (or best focus) $z_f$ and 
distance to the coverslip of the emitter $d_\text{cs} $ are not perfectly 
known. Therefore,
it is worth considering them as part of the optimization parameters that 
will be estimated along with the coefficients of the Zernike decomposition 
of the birefringent mask. The most obvious solution is to directly 
use $z_f$ and $d_{\text{cs}}$ as parameters, however both phase 
terms have similar effect when the 
radial coordinate $u$ is below the SAF radiation and this might cause the 
nonlinear optimization to fall into a local minimum which does not provide
the real values.
Moreover, the estimation 
of the distance of the best focus $z_f$ will generally depend on the distance 
of the nanobead from the coverslip. For example, if the paraxial 
approximation is used then the distance to the best focus will be 
$z_f = -n_f d_\text{cs} /n_i$. A similar result will hold even if we consider 
the spherical aberration induced by the interface, the only thing that changes 
is the factor in front of $d_\text{cs} $. Therefore, it is best to 
consider instead
\begin{align}
\bt G_0  (\bt u)  = D_{\delta, \alpha} (\bt u) \bt g_0 (\bt u),
\end{align}
where
\begin{multline}
D_{\delta, \alpha} (\bt u)= \exp \Bigg\{  
  \im 2 \pi n_f  \delta \Bigg[ \frac{n_i}{n_f} 
  \sqrt{1 - \left(\frac{n_f u}{n_i} \right)^2 }  \\
- \alpha \sqrt{1 - u^2} \Bigg] \Bigg\} , 
\end{multline}
with $\delta = d_\text{cs} / \lambda$ and $z_f = -\alpha d_\text{cs}$. 
Note that it has been assumed that the source is placed  
 along the optical axis so that $\bs \rho_0 = (0,0)$. 
Any deviation from this assumption will be corrected be the scalar tilts 
given by the tilt Zernike polynomials in the phase term $W$ of $\bt J_A$.
The parameter $\delta$ now predominantly controls the amount of exponential 
decay for the SAF radiation while $\alpha$ controls the defocus of the system. 
For fluorescing nanobead samples it is safe to 
assume that they are all fixed to the coverslip so that their distance to the 
coverslip can be taken as their radius and can be taken off the optimization
routine.  However, the choice of the best focus is quite subjective and should 
therefore always be included.


\subsection{Phase and polarization diversity}

In phase retrieval algorithms for optical microscopes, it is common 
practice to assume that we have access to a stack of intensity images 
at varying focal distances separated by $\Delta z_{\zeta}$ from the 
location of the best focus. This varying focal 
distances are taken into account by multiplying the Green tensor by 
the phase factor
\begin{align}
D^\zeta (\bt u) =  \exp \left[ \im k n_f \Delta z_\zeta  \sqrt{1 - u^2} \right] 
\end{align}
This additional information, referred to as phase diversity, helps the 
algorithm converge to an appropriate solution without falling into 
local minima and helps discriminate between the right and left phase vortices. 
These measurements are sufficient when the aberrations are taken as 
scalar, however when birefringence effects need to be taken into account 
it is necessary to implement a method to also provides information 
about the polarization state of the PSFs for each focal distance. 
This supplementary information can be obtained by inserting a polarization 
analyzer right after the 
birefringence mask (see Fig.~\ref{fig:setup}). This polarization analyzer 
can be composed of a combination of waveplates and polarizers where at least
one element can rotate in order to change the polarization projection of 
the output. This polarization diversity is modeled by a set of constant 
Jones matrices $\bt P^{(p)}$ that is applied after all other birefringent masks.
Therefore, the stack of Green tensors at the BFP is given by
\begin{align}
\bt G^{(\zeta ,p)}_\text{BFP} (\bt u) =  D^{(\zeta )} (\bt u) \bt P^{(p)} 
\cdot \bt J_\text{A}(\bt u) \cdot \bt J_\text{M}(\bt u) \cdot \bt G_0  (\bt u).
\end{align}
In this model $\bt J_\text{M}$ represents any known
birefringent mask into the nominal Green tensor which can have approximately known
parameters that can be incorporated into the optimization routine.
It is worth noting that while experimentally the polarization diversity happens at 
the BFP, computationally it is better to perform it at the image plane in 
order to avoid the computation of unnecessary DFTs.


\subsection{Modeling the total measured intensity}

To model the intensity measured by the camera, the Green tensor is first 
propagated to the image plane via,
\begin{align}
\bt G^{(\zeta ,p)}_\text{IP} (\bs \rho ) 
= \iint \bt G^{(\zeta ,p)}_\text{BFP} (\bt u)  
e^{ - \im k  \frac{n_f \bs \rho}{M}  \cdot \bt u } \ud \bt u .
\end{align}
Here, it will be assumed that the source emits fully unpolarized light. This 
is the case for fluorescing nanobeads which are commonly used to characterize or test 
fluorescence microscopes since they have a stronger signal than that emitted 
by single fluorescing molecules. In this case the measured intensity is 
given by the incoherent sum of the PSFs produced by dipoles oriented along 
each of the three Cartesian axes which is the same as the squared Frobenius 
norm of the Green tensor at the image plane 
\begin{align}
I^{(\zeta ,p)}_\text{IP} (\bs \rho ) = \norm{\bt G^{(\zeta ,p)}_\text{IP} }^2 
= \sum_{i=x,y} \sum_{j=x,y,z} |G^{(\zeta ,p)}_\text{IP,ij}( \bt u)|^2.
\end{align}
Depending on the size of the source it might be necessary to perform 
one of the blurring operations described in \cite{}. If a three-dimensional 
blurring is needed the computation of supplementary quantities will need to 
added to the forward model, for example, for the exact hard-sphere model 
we would also need to compute the total intensity for other values of $d_cs$.
The downside is that this would slow the algorithm considerably. 
However, as long as their 
diameter is smaller than 30nm one can safely skip this step.
More details about the blurring models implemented can be found in the 
Supplementary Information.

As a last step for computing the measured intensity, we also consider 
the effect of photobleaching of the fluorescing nanobeads and the background 
illumination. The photobleaching causes the number of photons emitted by the 
nanobead to diminish with time. Its effect can be taken into account by 
implementing an overall amplitude factor $\mathcal{A}^{(p,\zeta)}$
which depends on both the phase and polarization diversities. The background 
illumination is then added incoherently to the photobleached PSF stack. 
The simplest model is to assume that the background illumination is constant 
across each intensity image and determined by the term 
$\mathcal{B}^{(p,\zeta)}$. Therefore, the final total measured intensities 
of the ZP stack are given by 
\begin{align}
I^{(\zeta ,p)}_\text{tot} (\bs \rho )  
  =  a^{(p,\zeta)} I^{(\zeta ,p)}_\text{IP} (\bs \rho ) + b^{(p,\zeta)}.
\end{align}
It is possible to assume a more complicated model for the background 
illumination, such as a quadratic expansion \cite{aristov2018zola}. 

\subsection{Cost function}

The last piece we need to consider is the choice of a cost function that 
compares how good our modeled PSFs $I^{(\zeta ,p)}_\text{tot}$
are compared to the measured ones $I^{(\zeta ,p)}_\text{exp}$.
Under the absence of noise, any choice of cost function that has a minimum 
when the two quantities are the same should provide the same result. However,
noise is ever present in experimental measurements and thus needs to be taken
account. If the noise in the camera follows a Poisson distribution then the 
log-likelihood cost function,
\begin{align}
  \mathcal{C}_\text{LL} = \sum_{\zeta,p} \iint w(\bs \rho )\left\{ 
    I^{(\zeta ,p)}_\text{exp} (\bs \rho )
  \log \left[ I^{(\zeta ,p)}_\text{tot} (\bs \rho ) \right]
  - I^{(\zeta ,p)}_\text{tot} (\bs \rho )\right\} \ud^2 \bs \rho.
\end{align}
Whereas if it follows a Gaussian distribution then the sum of the squared 
difference is the most appropriate,
\begin{align}
  \mathcal{C}_\text{SS} = \sum_{\zeta,p} \iint w(\bs \rho )\left[
    I^{(\zeta ,p)}_\text{exp} (\bs \rho )
  - I^{(\zeta ,p)}_\text{tot} (\bs \rho )\right]^2 \ud^2 \bs \rho.
\end{align}
A window function $w$ has been introduced to represent the use of a finite
region at the image plane and to exclude bad pixels if there are any.
It is also worth making a technical note. In order for the choice of cost function 
to make sense we need to use as $I^{(\zeta ,p)}_\text{exp}$ the values that 
actually follows the assumed distribution. In general this means that the images
should not be denoised and that the offset of the camera should be removed.
A detailed description for the forward model can be found in the Supplementary 
Information. 


\section{Numerical experiments}

\subsection{Initial considerations}

To exemplify the implementation of the phase retrieval algorithm, 
let us consider the setup used for CHIDO. In this case the birefringent 
window is composed of a stress engineered optical (SEO) element followed by 
a quarter-wave plate at $45^{\circ}$. Afterwards, the light is separated into 
its $x$ and $y$ linear polarization components with a Wollaston prism which are 
then measured by the camera. This same setup can be used to obtain the necessary 
ZP stack for the phase retrieval algorithm by taking the SEO as the birefringent 
mask given by the Jones matrix
\begin{align}
\bt J_\text{SEO} (\bt u) = & \cos \frac{c u}{2}
\left( \begin{array}{cc}
1 & 0 \\
0 & 1 
\end{array}
\right)
+ \im \sin \frac{c u}{2}
\left( \begin{array}{cc}
\cos \phi & -\sin \phi \\
\sin \phi & - \cos \phi 
\end{array}
\right),
\end{align}
and the quarter-wave plate along with the Wollaston as the polarization analyzer.  To increase the number of polarization diversities, the quarter-wave plate will be allowed to be placed at various angles. This combination defines the polarization diversity Jones matrices through
\begin{align}
\bt P^{(p)} = \left\{ \begin{array}{c}
\left(
\begin{array}{cc}
1 & 0 \\
0 & 0 
\end{array}
\right) \\
\left(
\begin{array}{cc}
0 & 0 \\
0 & 1 
\end{array}
\right) 
\end{array}
\right\}
\cdot
{\begin{pmatrix}\cos ^{2}\theta_p +i\sin ^{2}\theta_p &(1-i)\sin \theta_p \cos \theta_p \\(1-i)\sin \theta_p \cos \theta_p &\sin ^{2}\theta_p +i\cos ^{2}\theta_p \end{pmatrix}}
\end{align}
where $\theta_p$ is the angle of the fast axis with respect to the $x$ axis.
Having defined the polarization diversity, the ZP stack can be acquired by 
taking images of the PSFs for all the polarization diversity at various 
focal planes. Note that before using these images for the retrieval of the 
aberrations and the other parameters, they need to be processed so that they 
follow the assumed statistics. This is done by converting the value at each 
pixel from the arbitrary gray scale value to the number of photocount events. 
In most cases, an offset value needs to be removed (if any pixel has a 
negative value then it should be set to zero), and then the gray value 
should be converted to the number of photoelectrons. 

Before being able to launch the phase retrieval, all the parameters that are 
going to be estimated need to be provided with an initial value. For the 
Zernike coefficients, the simplest choice is to set all equal to zero except 
the first coefficient of $A_0{(A)}$ which reduces the aberration matrix 
to the identity.
 
Depending on the particular type of sample used for acquiring the ZPstack, the 
distance to the coverslip might be estimated. For example, if the emitters are 
fixed at the surface of the coverslip then $d_\text{cs}$ is equal to the radius 
of the nanobead or if they are attached to a macro sphere, $d_\text{cs}$ can be 
estimated by the radial distance of the emitter from the center of the 
macrosphere. When $d_\text{cs}$ cannot be estimated from the sample, if the 
setup allows imaging the pupil $d_\text{cs}$ can be estimated from the 
exponential decay of the SAF radiation. Otherwise, an initial value not 
much greater to the nanobead's radius should work fine. 

For the parameter setting the distance of the best focus in terms of 
$d_\text{cs}$ there are several options. If, experimentally, there is a 
particular criteria to choose the best focus, e.g. the smallest rms width 
of the PSFs, then the same one could be used to define it. However, if the 
best focus is defined experimentally ``by eye'' then there are several 
sensible options, such as the parameter that  sets the second derivative 
of the defocus and SAF wavefront equal to zero or that the average of the 
first derivative is zero. It should be noted that in general this parameter 
should be greater than the one used for the paraxial focus sin the PSF 
quickly degrades as we move away from and closer to the coverslip but not 
when we move to the other side. 

The background illumination can be estimated directly form the images in 
the experimental ZP stack by averaging the pixels in the corners of each 
PSF image or in an empty region close to the nanobead being analyzed. 
Finally for the phobleaching coefficients, they could all be set equal 
to zero if the effect is negligible. Otherwise, they can be estimated 
numerically computing the PSFs, setting the scale of the computed PSFs so 
that the intensity of one of them is equal to the corresponding experimental 
image (the calculations should either include the background or it should be 
removed for the data just for his step), and then setting the photobleaching 
amplitudes equal to the ratio between the total intensity of the data and 
the modeled PSFs. 

As a lost point, it should be mentioned that the Jones matrix of the 
birefringent mask should be well characterized. In the case being treated 
here, this means that the radial parameter $c$ and the orientation set by 
the angle $\phi_0$ are known. They are measured from images of the pupil. 
Note that, if one desires, this parameters could be added to the phase 
retrieval algorithm by adding another branch in the forward and backward models.


\section{Characterization of }



%\section{Numerical simulations}
%
%\subsection{Test cases}
%
%\begin{itemize}
%\item Standard scalar aberrations
%\item Standard polarization and scalar aberrations
%\item S-plate
%\item Birefringent window (SEO)
%\end{itemize}
%
%\subsection{Scalar VS polarization aberrations}
%
%Our model allows us to turn off and on the coefficients used for the Zernike decomposition. In particular, the decomposition for the Jones matrix in terms of Pauli matrices allows us to consider purely scalar aberrations. Here, we explore how well purely scalar aberrations perform particularly when fitting high amounts of birefringent such as those used for PSF engineering. 
%
%
%\subsection{•}
%
%
%\section{Cross validation and the importance of polarization diversity}
%
%Due to experimental constraints we assumed that the source used for the characterization of the system emitted unpolarized light. This is in line with most common scenarios that use fluorescent nano-beads. However, when using the setup for a real measurement, the individual fluorophores emit light that depends on the amount of wobble. If the fluorophare stays fixed within  the acquisition time window, then it will emit light as a fully coherent dipole. 
%Given that the phase retrieval was done with unpolarized light, in order to test the accuracy of the model we run it on a cross-validation set of dipole with different orientations, at different focal planes and distance from the cover-slip. 
%





\begin{acknowledgments}
S. Paine for useful discussions.
\end{acknowledgments}

%% The Appendices part is started with the command
 \appendix

\section{Expressions for the Green tensor at the BFP}

As outlined in \cite{novotny2006principles}, a closed-form for the Green tensor at the BFP for a dipolar source placed close to an interface can be obtained. In particular the components of the $\bt g$  tensor in Eq.~(\ref{eq:green}) are given by
\begin{widetext}
\begin{align}
\bt g(\bt u) =\frac{1}{(1-u^2)^{1/4}} \left( 
\begin{array}{ccc}
\cos^2 \phi \sqrt{1-u^2} \Phi_2 + \sin^2 \phi \Phi_3 & \cos \phi \sin \phi ( \sqrt{1-u^2} \Phi_2 - \Phi_3) & -u \cos \phi \Phi_1  \\
\cos \phi \sin \phi ( \sqrt{1-u^2} \Phi_2 - \Phi_3) &  \sin^2 \phi \sqrt{1-u^2} \Phi_2 + \cos^2 \phi \Phi_3 & - u \sin  \phi \Phi_1
\end{array}
\right).
\end{align}
\end{widetext}
%\bse
%\begin{align}
%g_{11} =& \cos^2 \phi \sqrt{1-u^2} \Phi_2 + \sin^2 \phi \Phi_3  \\
%g_{12}=& \cos \phi \sin \phi ( \sqrt{1-u^2} \Phi_2 - \Phi_3) \\
%g_{13}=& -u \cos \phi \Phi_1 \\
%g_{21}=& g_{12} \\
%g_{22}=& \sin^2 \phi \sqrt{1-u^2} \Phi_2 + \cos^2 \phi \Phi_3 \\
%g_{23}=& - u \sin  \phi \Phi_1.
%\end{align}
%\ese
where 
\bse
\begin{align}
\Phi_1(u) = & t^p(u) \frac{n_i k_i\sqrt{1-u^2} }{n_m \sqrt{k_m^2 -k_i^2 u^2}} \\
\Phi_2(u) = & t^p(u)\frac{n_i}{n_m} \\
\Phi_3(u) = & t^s(u) \frac{k_i \sqrt{1-u^2}}{\sqrt{k_m^2 -k_i^2 u^2}}
\end{align}
\ese
with
\bse
\begin{align}
t^s(u) = & \frac{2k_{m,z}}{k_{m,z} + k_{i,z}},\\
t^p(u) = &  \frac{n_m}{n_i}\frac{2n_i^2 k_{m,z}}{n_i^2 k_{m,z} + n_m^2 k_{i,z}}.
\end{align}
\ese
being the Fresnel coefficients for p and s polarized light.


%\bibliography{Refs-raysNwaves}
\bibliography{microscopy}



\end{document}



